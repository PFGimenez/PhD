\documentclass[a4paper, 11pt]{article}

\usepackage[utf8]{inputenc}
\usepackage[T1]{fontenc}
\usepackage[english]{babel}
\usepackage{amsthm}
\usepackage{amsmath}
\usepackage{amssymb}
\usepackage{geometry}
\usepackage[bookmarks=true]{hyperref}
\usepackage[usenames,dvipsnames,svgnames,table]{xcolor}
\newenvironment{allintypewriter}{\ttfamily}{\par}
\title{User manual}
 
\author{Pierre-François \textsc{Gimenez}}
 
\date{version : \today}
 
\begin{document}

\maketitle

\section{Using interactive recommendation}

\texttt{java -jar interactive\_recom.jar algo dataset}

When a command has parameters, one should put every parameter on a new line.

The protocol is :

\texttt{exit} : exits the program. Nothing returned.

%\texttt{reinit [var]} : uninstantiates a variable. Nothing returned.

\texttt{reinit-all} : clears the current configuration session. Nothing returned.

\texttt{vars} : display the variables of the dataset. Returns a line with the variables name separated with commas.

\texttt{reco [var]} : asks for a recommendation. Returns three lines : the first contains the recommended value, the second the other possible value, the third the forbidden values. It is an error to ask a recommendation for a instantiated variable.

\texttt{isset [var]} : returns \texttt{true} if the variable is set, \texttt{false} otherwise.

\texttt{set [var] [value]} : affects the value to the variable (the value need to be possible). Nothing returned.

\texttt{value [var]} : returns the value of the variable (\texttt{null} if unset).

\texttt{value-all} : returns all the instantiated variables and their values.

\subsection{Example}

What has been prompt by the user is in black. The replies of the system is in red.

\begin{allintypewriter}
vars

\textcolor{red}{v1,v2,v3,v4,v5,v6,v7,v8,v9,v10,v11,v12,v13,v14,v15,v16,v17,\\v18,v19,v21,v23,v24,v25,v26,v27,v28,v29,v30,v31,v32,v33,v34,v35,\\v36,v37,v38,v39,v40,v41,v42,v43,v44,v45,v46,v93,v51,v52,v53}

reco

v6

\textcolor{red}{1}

\textcolor{red}{0}

\textnormal{\hspace{30pt}(empty line because there is no forbidden value)}

value

v6

\textcolor{red}{null}

set

v6

1

value

v6

\textcolor{red}{1}

reco

v1

\textcolor{red}{0}

\textcolor{red}{1}

\textcolor{red}{2}

isset

v1

\textcolor{red}{false}

set

v1

1

value

v1

\textcolor{red}{1}

isset

v1

\textcolor{red}{true}
\end{allintypewriter}


\end{document}